%==================================================
%      PREAMBOLO e DICHIARAZIONI INIZIALI
%==================================================
\documentclass[10pt,oneside,a4paper]{article}

\usepackage[latin1]{inputenc} 
\usepackage[italian]{babel}
\usepackage[T1]{fontenc}
\usepackage{siunitx} %Inserisce automaticamente i dati con le unit�  di misura correttamente formattate del SI (utilizzo: \SI{0.82}{m^2}, in generale \SI{misura con il punto decimale}{unit�  di misura})
\sisetup{output-decimal-marker = {.}, separate-uncertainty = true, input-uncertainty-signs = \pm, detect-weight=true, detect-family=true} %per usare SI con il punto decimale
\usepackage{listings} %Per citare codice informatico formattandolo correttamente
\usepackage{amsmath,amsthm,verbatim,amssymb,amsfonts,amscd,graphicx,mathtools}
\usepackage[makeroom]{cancel}
\newcommand{\abs}[1]{\left\lvert\,#1\,\right\rvert}
\usepackage{geometry}
\usepackage{epigraph}
\usepackage{booktabs}	%tabelle migliorate
\usepackage{tablefootnote}	%note a pi� di pagina in tabella
\usepackage{threeparttable} %tabella con note a pi� di tabella
\usepackage{caption}	%descrizione per figure
\usepackage{dblfnote}
\captionsetup{tableposition=top,figureposition=bottom,font=small} %setup descrizione
\usepackage{float}
\usepackage{esvect} %vettori
\usepackage{longtable} %tabelle lunghe
\usepackage[dvipsnames]{xcolor}
\definecolor{sepia}{HTML}{80002A}
\usepackage[colorlinks=true, citecolor=black, linkcolor=sepia, urlcolor=black]{hyperref}
\usepackage{mathrsfs}
\usepackage{circuitikz}


\usepackage{multicol}
\newenvironment{Figure}
  {\par\medskip\noindent\minipage{\linewidth}}
  {\endminipage\par\medskip}

\newcommand{\var}{\operatorname{var}}
\newcommand{\cov}{\operatorname{cov}}


\usepackage{listings} %Per inserire codice
\lstnewenvironment{codice_c}[1][]
{\lstset{basicstyle=\small\ttfamily, columns=fullflexible,
keywordstyle=\color{red}\bfseries, commentstyle=\color{blue},
language=C, basicstyle=\small,
numbers=left, numberstyle=\tiny,
stepnumber=2, numbersep=5pt, frame=shadowbox,  showstringspaces=false, #1}}{}

%==================================================
%                  PRIMA PAGINA
%==================================================

\title{\textsc{\textbf{Esercitazione 4}: Amplificatore operazionale 2}}
\author{\small{G. Galbato Muscio} \and \small{L. Gravina} \and \small{L. Graziotto}}
\date{6 Novembre 2018}

\begin{document}
	\begin{figure}
		\centering
		\includegraphics[scale=0.5, trim={2.8cm 8.9cm 0 9cm}, clip]{logo.png}
	\end{figure}
	\maketitle
	\begin{center} 
		\fbox{{\fontsize{12pt}{8mm}\textsc{Gruppo 11}}} \\
	\end{center}
\hrule
\vspace{0.5cm}
\renewcommand{\abstractname}{Abstract}
\begin{abstract}
Si realizzano un amplificatore differenziale e un sommatore non invertente utilizzando l'amplificatore operazionale LM358.
\end{abstract}
\vspace{4cm}
\tableofcontents %Indice
\newpage


\pagebreak
\begin{multicols}{2}
%==================================================
%      		AMPLIFICATORE DIFFERENZIALE
%==================================================
\section{Amplificatore differenziale}
Si utilizza nel seguito l'amplificatore operazionale LM358, e lo si alimenta con una differenza di potenziale di $\pm \SI{15}{V}$, connettendo gli ingressi \texttt{V+} e \texttt{GND} alla doppia alimentazione in continua, positiva e negativa. 

Si realizza un amplificatore differenziale con guadagno $G = 5$, come nel circuito seguente.

\begin{center}
\begin{circuitikz}
\draw (0,0) node[op amp] (opamp) {}
(opamp.+) to[R=$R_1'$] ++(-2,0) node[circ] {} node[left] {$v_\text{2}$}
(opamp.-) to[short] ++(0, 1.25) to[R = $R_2$] ++(2,0) -| (opamp.out)
(opamp.-) node[circ] {}
(opamp.+) node[circ] {}
(opamp.-) to[R=$R_1$] ++(-2,0) node[circ] {} node[left] {$v_1$}
(opamp.out) to[short, *-*] ++(0.5,0) node[right] {$v_\text{o}$}
(opamp.+) to[R=$R_2'$] ++(0,-2) node[ground] {}
;\end{circuitikz}
\end{center}

Scegliendo $R_1 = R_1'$ e $R_2 = R_2'$ si ha la relazione 
\[
v_o = \frac{R_2}{R_1}(v_2-v_1),
\]
ovvero il segnale in uscita � proporzionale alla differenza tra i segnali in ingresso. Pertanto si costruir� un circuito in cui i valori di $R_1$ e $R_1'$ e di $R_2$ e $R_2'$ siano il pi� simili possibile, e si sceglier� $R_2 / R_1 =: G = 5$. 

Le resistenze utilizzate, misurate con il multimetro, sono
\[
\begin{aligned}
R_1 &= \SI{111 \pm 111}{k\ohm} \\
R_1' &= \SI{111 \pm 111}{k\ohm} \\
R_2 &= \SI{111 \pm 111}{k\ohm} \\
R_2' &= \SI{111 \pm 111}{k\ohm}.
\end{aligned}
\]

Si misurano quindi l'amplificazione di modo comune $A_c$ e l'amplificazione differenziale $A_d$, nonch� il Common Mode Rejection Ratio (CMRR), $\rho = \abs{A_d / A_c}$, e li si confronta con i loro valori attesi dati i componenti del circuito e con quelli caratteristici di un amplificatore differenziale ideale.

Per misurare $A_c$ si invia lo stesso segnale ai due ingressi, ossia un segnale sinusoidale di frequenza $f$ e ampiezza $V_i = \SI{111 \pm 111}{V}$, che � abbastanza grande per osservare l'attenuazione data dal piccolo valore di $A_c$ e minore di \SI{111}{V}, valore oltre il quale si osservano distorsioni. Si riportano in tabella~\ref{tab:Ac_differenziale} le misure compiute a diverse frequenze. Il valor medio con associata la deviazione standard �
\[
A_c = 111 \pm 111.
\]


\begin{center}
\captionof{table}{Misure per la stima di $A_c$}
\label{tab:Ac_differenziale}
\begin{tabular}{c|c|c|c}
$f$ [\SI{}{kHz}] & $V_i$ [\SI{}{V}] & $v_o$ [\SI{}{V}] & $A_c = v_o / V_i$ \\
\hline
\hline
\end{tabular}
\end{center}

Per misurare $A_d$ si devono invece ricavare singolarmente le amplificazioni $A_1$ e $A_2$, inviando un solo segnale ad un ingresso e mettendo a terra l'altro. Inviando il segnale a $v_1$ e mettendo a terra $v_2$ si avr� un amplificatore invertente, dunque ci si aspetta di ottenere 
\[
A_1 = -\frac{R_2}{R_1}.
\]
Inviando il segnale a $v_2$ e mettendo a terra $v_1$, invece, si avr� un amplificatore non invertente, dunque ci si aspetta di ottenere 
\[
A_2 = \frac{R_2'}{R_1}\frac{R_1+R_2}{R_1'+R_2'}.
\] 

Si invia dunque alternativamente ad un ingresso e poi all'altro (mettendo a terra quello non usato) un segnale sinusoidale di ampiezza $v_{1,2}$ e frequenza $f$, e si misura l'uscita $v_o$ sul canale \texttt{CH2} dell'oscilloscopio. Quindi si ricavano $A_1 = v_o / v_{1,2}$ e $A_2 = v_o / v_{1,2}$ e l'amplificazione differenziale $A_d = (A_1 - A_2) / 2$. Si riportano in tabella~\ref{tab:Ad_differenziale} le misure compiute a diverse frequenze.

\begin{table*}
\begin{center}
\captionof{table}{Misure per la stima di $A_d$}
\label{tab:Ad_differenziale}
\begin{tabular}{c|c|c|c|c|c|c}
$f$ [\SI{}{kHz}] & $v_{1,2}$ [\SI{}{V}] & $v_o$ (con $v_2 = 0$) [\SI{}{V}] & $v_o$ (con $v_1 = 0$) [\SI{}{V}] & $A_1$ & $A_2$ & $A_d$\\
\hline
\hline
\end{tabular}
\end{center}
\end{table*}

Si hanno come valori medi delle amplificazioni di singolo canale 
\[
\begin{aligned}
A_1 &= 111 \pm 111 \\
A_2 &= 111 \pm 111
\end{aligned}
\]
compatibili con i valori attesi dati i componenti usati $A_1 = 111 \pm 111$ e $A_2 = 111 \pm 111$. Il valore medio dell'amplificazione differenziale, con associata incertezza data dalla deviazione standard, �
\[
A_d = 111 \pm 111,
\]
compatibile con il valore atteso dati i componenti di $A_d = 111 \pm 111$. Si pu� confrontare anche il valore misurato di $A_c$ con quello atteso $A_c = A_1 + A_2 = 111 \pm 111$, osservando che essi sono compatibili.

Si nota inoltre che il valore di $A_c$ � prossimo allo zero, mentre quello di $A_d$ al valore pari al guadagno $G = 5$, secondo il quale sono state scelte le resistenze all'inizio.

Si stima quindi il CMRR 
\[
\rho = \abs{\frac{A_d}{A_c}} = 111 \pm 111,
\]
compatibile con il valore atteso di $\rho = 111 \pm 111$. Si osserva che questo valore � molto elevato, infatti nell'amplificatore differenziale ideale si dovrebbe avere $\rho \rightarrow \infty$.

%==================================================
%    		SOMMATORE NON INVERTENTE
%==================================================
\section{Sommatore non invertente}
Si realizza un circuito sommatore non invertente a due ingressi, come indicato nella figura seguente.

\begin{center}
\begin{circuitikz}
\draw (0,0) node[op amp] (opamp) {}
(opamp.out) to[short, *-*] ++(0.5,0) node[right] {$V_\text{o}$}
(opamp.-) to[short] ++(-0.25,0) to[R=$R$] ++(-1.5,0) node[ground] {}
(opamp.-) -| ++(0,1) to[R = $R'$] ++(2,0) -| (opamp.out)
(opamp.-) node[circ] {}
(opamp.+) to[R=$R_A$] ++(-2,0) node[circ] {} node[left]{$V_A$}
(opamp.+) node[circ] {}
(opamp.+) to[short] ++(0,-1) to[R=$R_B$] ++(-2,0) node[circ] {} node[left]{$V_B$}
;\end{circuitikz}
\end{center}

I valori delle resistenze utilizzate, misurati con il multimetro, sono
\[
\begin{aligned}
R_A &= \SI{111 \pm 111}{k\ohm} \\
R_B &= \SI{111 \pm 111}{k\ohm} \\
R &= \SI{111 \pm 111}{k\ohm} \\
R' &= \SI{111 \pm 111}{k\ohm}; 
\end{aligned}
\]

si sono scelte resistenze tutte simili tra loro in modo che valga la relazione
\[

\]




\end{multicols}
\end{document}